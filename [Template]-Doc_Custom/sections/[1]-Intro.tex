\section{Instrucciones}
\subsection{Subtitulo}
\subsubsection{Subsubtitulo}
Borrar todos los \verb|lipsum| en el código antes de escribir el contenido.

\textbf{asd}

Para citar utilice los siguientes comandos (Ver en código): \autocite{ejemplo1} y \autocite{ejemplo2} y \autocite{Parac}.


\section{Metodología}
\lipsum[2]\\
\highlight{2cm}{\lipsum[75]}


\section{Resultados y discusión}
\lipsum[75]


\section{Conclusiones}
\lipsum[2] Figure \ref{ref:tabla}

\lipsum[4]


\clearpage


\section{Ejemplos adicionales}
\begin{figure}[H]
    \centering
    \begin{tikzpicture}
        \begin{axis}[clip=false, xmin=-5, xmax=5, ymin=-10, ymax=10, axis lines=middle, xlabel=$x$, ylabel=$y$]
            \addplot[red, samples=50, name path=f, dashed, domain=-3:4]{x^2}
            node[right,pos=1]{$f(x)=x^2$};
            \addplot[blue, samples=50, domain=-2:2, name path=g]{1-x^2}
            node[right,pos=1]{$g(x)=1-x^2$};
            \addplot[green, opacity=0.2] fill between[of=f and g, soft clip={domain=-0.70:0.70}];
        \end{axis}
    \end{tikzpicture}
    \caption{Titulo}
    \label{ref:tabla}
\end{figure}

\begin{align*}
    f(x) &= x^2\\
    g(x) &= 1-x^2\\
    &= 1-2x^2
    % This aligns the = signs
\end{align*}

\begin{table}[H]
    \centering
    \begin{tabular}{c|cc}
        asd & asd & asd \\
        \hline
        asd & asd & asd
    \end{tabular}
    \caption{Caption}
    \label{tab:my_label}
\end{table}